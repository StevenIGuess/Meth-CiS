\documentclass[a4paper,tikz]{article}
\usepackage[utf8]{inputenc}
\usepackage[ngerman]{babel} %Silbentrennung und Sprache: deutsch
\usepackage{lmodern}        %Schrift (Suche in PDF nach Umlautwörtern)
\usepackage[T1]{fontenc}    %Silbentrennung bei Umlauten (mit lmodern)
\usepackage[a4paper,top=2cm,bottom=2cm,left=3cm,right=3cm,marginparwidth=1.75cm]{geometry}
 \usepackage[outline]{contour} % glow around text
\usepackage{amsmath,amssymb,amsthm,bbm} %AMS-Pakete
\usepackage{tikz}
\usepackage{physics}
\usepackage{mathtools}

\usetikzlibrary{angles,quotes} % for pic
\usetikzlibrary{bending} % for arrow head angle
\contourlength{1.0pt}



\tikzset{>=latex} % for LaTeX arrow head
\usepackage{xcolor}
\colorlet{myblue}{blue!65!black}
\colorlet{mydarkblue}{blue!50!black}
\colorlet{myred}{red!65!black}
\colorlet{mydarkred}{red!40!black}
\colorlet{veccol}{green!70!black}
\colorlet{vcol}{green!70!black}
\colorlet{xcol}{blue!85!black}
\colorlet{mypurple}{blue!50!red!90!black!60}
\tikzstyle{vector}=[->,very thick,line cap=round]
\tikzstyle{xline}=[myblue,very thick]
\tikzstyle{yzp}=[canvas is zy plane at x=0]
\tikzstyle{xzp}=[canvas is xz plane at y=0]
\tikzstyle{xyp}=[canvas is xy plane at z=0]
\def\tick#1#2{\draw[thick] (#1) ++ (#2:0.12) --++ (#2-180:0.24)}
\def\N{100}

\newcommand\IVeq{\stackrel{\mathclap{\normalfont\mbox{iv}}}{=}}
\newcommand\EXeq{\stackrel{\mathclap{\normalfont\mbox{!}}}{=}}
\newcommand*{\QED}{\null\nobreak\hfill\ensuremath{\square}}%
\newcommand{\crel}[1]{%
  \global\setbox1=\hbox{$#1$}%
  \global\dimen1=0.5\wd1
  \mathrel{\hbox to\dimen1{$#1$\hss}}&\mathrel{\mspace{-\thickmuskip}\hbox to\dimen1{}}%
}

\title{Blatt03}
\author{Toma-Stefan Cezar (Matr. 7678219), Hai-Yen Van (Matr. 7611734),\\ Thuy An Le (Matr. 7510768)}
\date{November 2022}

\begin{document}

\maketitle
Neue Abgabegruppen!
\tableofcontents
\newpage

\section{Aufgabe 1}

\subsection{Teil A}

\[\frac{5}{(2-i)^2} = \frac{5}{4-4i-1} = \frac{5(3+4i)}{(3-4i)(3+4i)} = \frac{15 + 20i}{25} = \frac{3}{5} + \frac{4}{5}i\]
\[ (1+i)^6-(i-1)^6 = ((1+i)^2)^3 - ((1-i)^2)^3 = (1 + 2i - 1)^3 - (1-2i-1)^3 = (2i)^3-(-2i)^3 = 16i^3 = 0 - 16i\]
\[ \frac{1+2i}{4-(2+1)^2} = \frac{1+2i}{4-(4+4i-1)} = \frac{1+2i}{1-4i} = \frac{(1+2i)(1+4i)}{(1-4i)(1+4i)} = \frac{1-8 + 4i+2i}{1+16} = -\frac{7}{17} + \frac{6}{17}i\]


\begin{center}
\begin{tikzpicture}

  \def\x{5}
  \def\y{4}
  \def\R{1.9}
  \def\ang{35}
  \coordinate (O) at (0,0);
  
  %all values are scaled by 2 except 16
  
  \coordinate (A) at (1.2,1.6);
  \coordinate (B) at (0,-3.8);
  \coordinate (C) at (-0.824,0.706);
  
  
  \coordinate (AX) at (1.2,0);
  \coordinate (AY) at (0,1.6);
  \coordinate (BY) at (0,-3.8);
  \coordinate (CX) at (-0.824,0);
  \coordinate (CY) at (0,0.706);
  
  
  \draw[->,line width=0.9] (-\x,0) -- (\x,0) node[right] {Re};
  \draw[|->,line width=0.9] (0,-3) -- (0,\y) node[left] {Im};
  \draw[-|,line width=0.9] (0,-4) -- (0,-3.08);
  
  \node[myblue,above right=-2] at (A) {$\frac{3}{5} + \frac{4}{5}i$};
  \node[myblue,above right=2] at (B) {$0 -16i$};
  \node[mypurple,above left=2] at (C) {$-\frac{7}{17} + \frac{6}{17}i$};
  \draw[dashed,myblue] (AY) -- (A);
  \draw[dashed,myblue] (AX) -- (A);
  \draw[dashed,myred] (CY) -- (C);
  \draw[dashed,myred] (CX) -- (C);
  \draw[vector, myblue] (O) -- (A) node[pos=0.55,above left=-2] {};
  \draw[vector, mypurple] (O) -- (B) node[pos=0.55,above left=-2] {};
  \draw[vector, myred] (O) -- (C) node[pos=0.55,above left=-2] {};
  \tick{AX}{90}[myblue] node[myblue,scale=0.9,below=-1] {$\frac{3}{5}$};
  \tick{AY}{ 0}[myblue] node[myblue,scale=0.9,left] {$\frac{4}{5}$};
  \tick{BY}{ 0}[mypurple] node[mypurple,scale=0.9,left] {$-16$};
  \tick{CX}{90}[myred] node[myred,scale=0.9,below=-1] {$-\frac{7}{17}$};
  \tick{CY}{ 0}[myred] node[myred,scale=0.9, right=3.5] {$\frac{6}{17}$};

\end{tikzpicture}
\end{center}
\newpage

\subsection{Teil B}

\[z^2 = 4i\]
\[(a+bi)^2 = 4i\]
\[4i = a^2 + 2abi - b^2\]
Aus $\operatorname{Re}(4i) = 0$ folgt,
\[a^2-b^2 = 0\]
\[a^2 = b^2\]
\begin{equation}
 \pm a = \pm b
\end{equation}
Aus $\operatorname{Im}(4i) = 4$ folgt,
\[2abi = 4i \]
\[ ab = 2\]
\[ a \cdot a \stackrel{\mathclap{\normalfont\mbox{(1)}}}{=} 2 \]
\[ a_{1,2} = b_{1,2} = \pm \sqrt{2}\]
Daraus folgt nun,
\[z_1 = \sqrt{2} + \sqrt{2}i\]
\[z_2 = -\sqrt{2} - \sqrt{2}i\]



\subsection{Teil C}

\[ z^3 +2z^2 + 2z = 0\]
\[ z(z^2 +2z + 2) = 0\]
Durch Anwenden der Nullproduktregel, erhält man:
\[z_1 = 0\]
\[z^2 +2z + 2 = 0\]
Nun kann man in die allgemeine quadratische Gleichung (a-b-c-Formel) einsetzen:
\[z_{2/3} = \frac{-2 \pm \sqrt{2^2 - 4 \cdot 1 \cdot 2}}{2 \cdot 1} = \frac{-2 \pm \sqrt{-4}}{2} = \frac{-2}{2} \pm \frac{\sqrt{-4}}{2} = -1 \pm i\]
Die komplexen Lösungen sind $z_1 = 0$, $z_2 = -1 - i$, $z_3 = -1 + i$.

\section{Aufgabe 2}
\subsection{Teil A}
\subsubsection{Teil 1}

\[z^4 = (a+bi)^4 = \big((a+bi)^2\big)^2\]
\[= (a^2+2abi-b^2)^2\]
\[= a^4+4a^3bi-6a^2b^2-4b^3i+b^4\]
\[= (a^4-6a^2b^2+b^4) + (4a^3b-4ab^3)i\] \\
\[ \operatorname{Re}(z^4) = (a^4-6a^2b^2+b^4)\]
\[ \operatorname{Im}(z^4) = (4a^3b-4ab^3)\]
\newpage
\subsubsection{Teil 2}
\[\frac{1}{z^2} = \frac{1}{(a+bi)^2} = \frac{1}{(a+bi)(a+bi)}\]
\[= \frac{1}{a+bi} \cdot \frac{1}{a+bi} = \frac{(a-bi)}{(a+bi)(a-bi)} \cdot \frac{(a-bi)}{(a+bi)(a-bi)} = \frac{(a-bi)(a-bi)}{(a^2+b^2)(a^2+b^2)}\]
\[= \frac{(a-bi)^2}{(a^2+b^2)^2} = \frac{a^2 - 2abi - b^2}{(a^2+b^2)^2}\]
\[= \frac{a^2-b^2}{(a^2+b^2)^2} + \frac{-2ab}{(a^2+b^2)^2}i\]
\\
\[ \operatorname{Re}(\frac{1}{z^2}) = \frac{a^2-b^2}{(a^2+b^2)^2}\]
\[ \operatorname{Im}(\frac{1}{z^2}) = \frac{-2ab}{(a^2+b^2)^2}\]

\subsection{Teil B}
Unter der Annahme $\forall z \in \mathbb{C}: z = x+yi, x \in \mathbb{R}, y \in \mathbb{R}$,
\[|z| \leq |\operatorname{Re}(z)| + |\operatorname{Im}(z)| \Longleftrightarrow \sqrt{x^2 + y^2} \leq |x| + |y|\]
\[\Longleftrightarrow x^2 + y^2 \leq (|x| + |y|)^2\]
Es gilt $|x^2| = \sqrt{x^2}^2 = x^2$,
\[\Longleftrightarrow x^2 + y^2 \leq x^2 + |2xy| + y^2\]
\[\Longleftrightarrow 0 \leq |2xy| \]
Diese Aussage stimmt, da der Betrag einer Zahl immer positiv oder Null ist. QED\\
\QED

\begin{center}
\begin{tikzpicture}

  \def\x{3}
  \def\y{3}
  \def\R{1.9}
  \def\ang{35}
  \coordinate (O) at (0,0);
  
  %all values are scaled by 2 except 16
  
  \coordinate (A) at (1.5,1.5);
  
  
  \coordinate (AX) at (1.5,0);
  \coordinate (AY) at (0,1.5);

  
  \draw[->,line width=0.9] (-1,0) -- (3,0) node[right] {Re};
  \draw[->,line width=0.9] (0,-1) -- (0,3) node[left] {Im};
  
  \node[myblue,above right=-2] at (A) {$z(x,y)$};
  \draw[dashed,myblue] (AY) -- (A) node[pos=0.5, above] {$|\operatorname{Re}(z)|$};
  \draw[dashed,myblue] (AX) -- (A) node[pos=0.5, rotate=90, below] {$|\operatorname{Im}(z)|$};
  \draw[vector, myblue] (O) -- (A) node[pos=0.55,above left=-2] {};

\end{tikzpicture}
\end{center}

\newpage

\section{Aufgabe 3}
\subsection{Teil A}


\subsubsection{Das neutrale Element}

Laut der Tabelle, welche die Verknüpfung $+$ definiert:
\[a + a = a\]
\[b + a = b\]
\[c + a = c\]
Somit ist $a$ das neutrale Element der Verknüpfung $+$ in $\mathbb{F}_3$.


\subsubsection{Die additiven Inversen}

\[\forall x \in \mathbb{F}_3 \exists -x \in \mathbb{F}_3: x+(-x) = a\]
Laut der Tabelle, welche die Verknüpfung $+$ definiert:
\[a = a + a\]
\[a = b + c\]
\[a = c + b\]
Somit sind die additiven Inversen $-x$,
\begin{center}
\begin{tabular}{ c|c } 
 $x$ & $-x$ \\
 \hline
 $a$ & $a$ \\
 $b$ & $c$ \\
 $c$ & $b$ \\
\end{tabular}
\end{center}



\subsection{Teil B}
\subsubsection{Das neutrale Element}
Laut der Tabelle, welche die Verknüpfung $\cdot$ definiert:
\[a \cdot b = a\]
\[b \cdot b = b\]
\[c \cdot b = c\]
Somit ist $b$ das neutrale Element der Verknüpfung $\cdot$ in $\mathbb{F}_3$.

\subsubsection{Die multiplikativen Inversen}
\[\forall x \in \mathbb{F}_3 \exists x^{-1} \in \mathbb{F}_3 : x \cdot x^{-1} = b\]
Laut der Tabelle, welche die Verknüpfung $\cdot$ definiert\footnote{Das neutrale Element von $+$ hat kein multiplikatives Inverses}:
\[b = b \cdot b\]
\[b = c \cdot c\]
Somit sind die multiplikativen Inversen $x^{-1}$,
\begin{center}
\begin{tabular}{ c|c } 
 $x$ & $x^{-1}$ \\
 \hline
 $b$ & $b$ \\
 $c$ & $c$ \\
\end{tabular}
\end{center}


\subsection{Teil C}
\[\forall y,z \in \mathbb{F}_3 : c \cdot (y+z) = cy+cz = (y+z) \cdot c\]
Die Gleichung $c(y+z) = (y+z) \cdot c$, stimmt da die Verknüpfungen $+,\cdot$ in $\mathbb{F}_3$ kommutativ sind. Die Gleichung $c(y+z) = cy + cz$, lässt sich durch diese Tabelle beweisen:

\begin{center}
\begin{tabular}{ c|c|c|c|c|c|c|c } 
 $y$ & $z$ & $y+z$ & $c(y+z)$ & $cy$ & $cz$ & $cy + cz$ & $c(y+z) \Longleftrightarrow cy + cz$ \\
 \hline
 $a$ & $a$ & $a$ & $a$ & $a$ & $a$ & $a$ & $w$ \\
 $a$ & $b$ & $b$ & $c$ & $a$ & $c$ & $c$ & $w$ \\
 $a$ & $c$ & $c$ & $b$ & $a$ & $b$ & $b$ & $w$ \\
 $b$ & $a$ & $b$ & $c$ & $c$ & $a$ & $c$ & $w$ \\
 $b$ & $b$ & $c$ & $b$ & $c$ & $c$ & $b$ & $w$ \\
 $b$ & $c$ & $a$ & $a$ & $c$ & $b$ & $a$ & $w$ \\
 $c$ & $a$ & $c$ & $b$ & $b$ & $a$ & $b$ & $w$ \\
 $c$ & $b$ & $a$ & $a$ & $b$ & $c$ & $a$ & $w$ \\
 $c$ & $c$ & $b$ & $c$ & $b$ & $b$ & $c$ & $w$ \\
\end{tabular}
\end{center}

 QED.\\
 \QED

\end{document}


 
