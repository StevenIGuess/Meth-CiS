\documentclass[a4paper]{article}
\usepackage[utf8]{inputenc}
\usepackage[ngerman]{babel} %Silbentrennung und Sprache: deutsch
\usepackage{lmodern}        %Schrift (Suche in PDF nach Umlautwörtern)
\usepackage[T1]{fontenc}    %Silbentrennung bei Umlauten (mit lmodern)
 \usepackage[a4paper,top=2cm,bottom=2cm,left=3cm,right=3cm,marginparwidth=1.75cm]{geometry}
 
\usepackage{amsmath,amssymb,amsthm,bbm} %AMS-Pakete

\usepackage{mathtools}

\newcommand\IVeq{\stackrel{\mathclap{\normalfont\mbox{iv}}}{=}}
\newcommand\EXeq{\stackrel{\mathclap{\normalfont\mbox{!}}}{=}}
\newcommand*{\QED}{\null\nobreak\hfill\ensuremath{\square}}%
\newcommand{\crel}[1]{%
  \global\setbox1=\hbox{$#1$}%
  \global\dimen1=0.5\wd1
  \mathrel{\hbox to\dimen1{$#1$\hss}}&\mathrel{\mspace{-\thickmuskip}\hbox to\dimen1{}}%
}

\title{Blatt02}
\author{Toma-Stefan Cezar (Matr. 7678219), Elham Amini (Matr. 7606587)}
\date{November 2022}

\begin{document}

\maketitle
Neue Abgabegruppen!
\tableofcontents
\newpage

\section{Aufgabe 1A}

Beweisen Sie durch vollständige Induktion:
\[ \forall n \in \mathbb{N},n \geq 1 :  \sum_{k=1}^n k(k+1) = \frac{1}{3}n(n+1)(n+2)\]

\subsection{Induktionsanfang}
Die Aussage gilt für $n=1$, da
\[ \sum_{k=1}^1 k(k+1) = \frac{1}{3} (1 + 1)(1 + 2) \]
\[ 1(1+1) = \frac{6}{3} \]
\[ 2 = 2 \]

\subsection{Induktionsschritt}

\[ z.z:  \sum_{k=1}^{n+1} k(k+1) \EXeq \frac{1}{3}(n+1)(n+2)(n+3) \]
\newline 
Angenommen die Induktionsvoraussetzung iv $ : \sum_{k=1}^n k(k+1) = \frac{1}{3}n(n+1)(n+2)$ stimmt.

\[ \sum_{k=1}^{n+1} k(k+1) = (\sum_{k=1}^{n} k(k+1)) + (n+1)(n+2) \IVeq \frac{1}{3}n(n+1)(n+2) + (n+1)(n+2)\] 
\[ = \frac{1}{3}n(n+1)(n+2) + (n+1)(n+2) \]
\[ = (n+1)(n+2)(\frac{1}{3}n+1)\]
\[ = \frac{1}{3}(n+1)(n+2)(n+3)\] 

Somit ist der Induktionsschritt $\sum_{k=1}^{n+1} k(k+1) = \frac{1}{3}(n+1)(n+2)(n+3)$ beweisen. QED
\QED

\newpage

\section{Aufgabe 1B}

Beweisen Sie durch vollständige Induktion:
\[ \forall n \in \mathbb{N},n \geq 4 :  2n \leq n^2 - 1 \leq 2^n -1\]

\subsection{Teil 1}

\subsubsection{Induktionsanfang}
Die Aussage gilt für $n=4$, da
\[2 \cdot 4 \leq 4^2 - 1 \Longleftrightarrow  8 \leq 15\]

\subsubsection{Induktionsschritt }

\[ z.z:  2(n+1) \leq (n+1)^2 - 1 \]
\newline 
Angenommen die Induktionsvoraussetzung iv $ 2n \leq n^2 - 1 $ stimmt.

\[ 2(n+1) \leq (n+1)^2 - 1\]
\[ \Longleftrightarrow 2n+2 \leq n^2 + 2n\]
\[ \IVeq (n^2-1)+2 \leq n^2 + 2n\]
\[ \Longleftrightarrow n^2+1 \leq n^2 + 2n\]
\[ \Longleftrightarrow 1 \leq 2n\]
\\
Diese Aussage ist erfüllt, da $2n$ für $n \geq 4$ immer größer ist.


\subsection{Teil 2}

\subsubsection{Induktionsanfang}
Die Aussage gilt für $n=4$, da
\[ 4^2 - 1 \leq 2^4 -1 \Longleftrightarrow 15 \leq 15\]

\subsubsection{Induktionsschritt }

\[ z.z: (n+1)^2 - 1 \leq 2^{n+1} - 1 \]
\newline 
Angenommen die Induktionsvoraussetzung iv $ n^2 - 1 \leq 2^n -1 \Longrightarrow n^2 \leq 2^n$ stimmt.
\[ (n+1)^2 - 1 \leq 2^{n+1} - 1 \]
\[ \Longleftrightarrow n^2+2n \leq 2^n \cdot 2^1 - 1 \]
\[ \IVeq n^2 \leq 2n^2 - 1 \]
\[ \Longleftrightarrow n^2 + 2n + 1 \leq 2n^2\]
\[ \Longleftrightarrow 2n + 1 \leq n^2\]\\
Durch Ableiten erhält man.
\[ \frac{d}{dn} 2n + 1 = 2 \]
\[ \frac{d}{dn} n^2 = 2n \]
\\
Anhand dieser Gleichungen lässt sich erkennen, dass $n^2$ für $n>1$ schneller wächst als $2n + 1$, deswegen gilt die Aussage  $(n+1)^2 - 1 \leq 2^{n+1} - 1$ für alle $n \geq 4$.

\subsubsection{Zusammenfassung}
Da 2.1 und 2.2 bewiesen sind, ist auch $\forall n \in \mathbb{N},n \geq 4 :  2n \leq n^2 - 1 \leq 2^n -1$ bewiesen. QED 
\QED




\newpage

\section{Aufgabe 2}

Beweisen Sie durch vollständige Induktion:
\[ \forall m \in \mathbb{N},m \geq 1 :  \sum_{j=0}^m (2j+1)(2j+3) = \frac{1}{3}(m+1)(4m^2 + 14m + 9)\]

\subsection{Induktionsanfang}
Die Aussage gilt für $m=1$, da


\[\sum_{j=0}^1 (2j+1)(2j+3)  = \frac{1}{3}(1+1)(4 \cdot (1)^2 + 14 \cdot 1 + 9)\]

\[ (2 \cdot 0 + 1)(2 \cdot 0 + 3) + (2 \cdot 1 + 1)(2 \cdot 1 + 3) =  \frac{1}{3}(1+1)(4 \cdot (1)^2 + 14 \cdot 1 + 9) \]
\[ (1 \cdot 3) + (3 \cdot 5) =  \frac{1}{3} \cdot 2 \cdot (4 + 14 + 9) \]
\[ 18 = \frac{1}{3} \cdot 54\]
\[ 18 = 18 \]

\subsection{Induktionsschritt}

\begin{equation}
 z.z:  \sum_{j=0}^{m+1} (2j+1)(2j+3) \EXeq \frac{1}{3}((m+1) + 1)(4(m+1)^2 + 14(m+1) + 9) 
\end{equation}
\[ = \frac{1}{3}(m+2)(4((m+1)(m+1)) + 14(m+1) + 9) \]
\[ = \frac{1}{3}(m+2)(4((m^2 + 2m + 1^2) + 14m + 14 + 9) \]
\[ = \frac{1}{3}(m+2)(4m^2 + 8m + 4 + 14m + 23) \]
\[ = \frac{1}{3}(4m^3 + 8m^2 + 4m + 14m^2 + 23m + 8m^2 + 16m + 8 + 28m + 46) \]
\[ = \frac{1}{3}(4m^3 + 30m^2 + 71m + 54) \]
\newline 
Angenommen die Induktionsvoraussetzung iv $ : \sum_{j=0}^m (2j+1)(2j+3) = \frac{1}{3}(m+1)(4m^2 + 14m + 9)$ stimmt.


\begin{equation}
\sum_{j=0}^{m+1} (2j+1)(2j+3) = \sum_{j=0}^m (2j+1)(2j+3) + (2(m+1)+1)(2(m+1)+3)
\end{equation}
\[ \IVeq  \frac{1}{3}(m+1)(4m^2 + 14m + 9) + (2(m+1)+1)(2(m+1)+3) \]
\[ = \frac{1}{3} \cdot \big((m+1)(4m^2 + 14m + 9)\big) + (2m+3)(2m+5)\]
\[ = \frac{1}{3} \cdot \big((m+1)(4m^2 + 14m + 9)\big) + (2m)^2 + 10m + 6m + 15\]
\[ = \frac{1}{3} \cdot (4m^3 + 14m^2 + 9m + 4m^2 + 14m + 9) + 4m^2 + 10m + 6m + 15\]
\[ = \frac{(4m^3 + 14m^2 + 9m + 4m^2 + 14m + 9)}{3}  + \frac{3 \cdot (4m^2 + 10m + 6m + 15)}{3}\]
\[ = \frac{4m^3 + 14m^2 + 9m + 4m^2 + 14m + 9 + 12m^2 + 48m + 45)}{3}\]
\[ = \frac{1}{3}(4m^3 + 30m^2 + 71m + 54)\]
\endline
Da (1) und (2) gleich sind ist der Induktionsschritt bewiesen. QED \QED


\newpage
\section{Aufgabe 3A}
Seien $A,B,C,D$ Mengen und $f: A \rightarrow B$, $g: B \rightarrow C$, $h: C \rightarrow D$ Abbildungen, dann definieren die Verknüpfungen $(h \circ g) \circ f$ und $h \circ (g \circ f)$ beide die Funktion $h(g(f(x)))$ bzw. die Abbildung $h \circ g \circ f : A \rightarrow D$.

\[ ((h \circ g) \circ f)(x) = ( h \circ g)(f(x)) = h(g(f(x))) \]
\[ (h \circ (g \circ f))(x) = h((g \circ f ))(x = h(g(f(x))) \]
\[ h \circ g \circ f : A \rightarrow D, x \mapsto (h \circ g \circ f) := h(g(f(x)))\]
\newline
Da beide Verknüpfungen die selbe Abbildung ergeben, ist das Verknüpfen von Abbildern assoziativ. QED
\newline
\QED

\section{Aufgabe 3B}
Beweis durch Gegenbeispiel: 
Angenommen $f: A \rightarrow B$ ist surjektiv, so muss jedes Element von $A$ einem Element aus $B$ zugeordnet sein $\forall a \in A \exists b \in B : f(a) = b$. Die Abbildung $g : B \rightarrow A$ ordnet nun jedem Element aus $B$ mindestens ein Element aus $A$ zu, daher ist $(f \circ g) : B \rightarrow B = \mathrm{Id}_{B}$. Wenn $f$ nicht surjektiv ist, muss es mindestens ein Element in $B$ geben, zu welchem kein Element aus $A$ zugeordnet werden kann, somit kann das Abbild $g : B \rightarrow A $ nicht existieren, da es mindestens einem Wert aus $B$ keinen Wert zuweisen kann. QED 
\newline
\QED

\end{document}





 
