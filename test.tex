\documentclass[a4paper]{article}
\usepackage[utf8]{inputenc} %Umlaute können direkt verwendet werden
\usepackage[ngerman]{babel} %Silbentrennung und Sprache: deutsch
\usepackage{lmodern}        %Schrift (Suche in PDF nach Umlautwörtern)
\usepackage[T1]{fontenc}    %Silbentrennung bei Umlauten (mit lmodern)
 
\usepackage{amsmath,amssymb,amsthm,bbm} %AMS-Pakete
\theoremstyle{definition}               %Definitionen nicht "kursiv"
\newtheorem{bemerkung}{Bemerkung}
\newtheorem{problem}{Aufgabe}
\theoremstyle{plain}                    %Sätze etc. "kursiv"
\newtheorem{satz}{Satz}
\newtheorem{lemma}{Lemma}
\newenvironment{beweis}[1][Beweis]{\begin{proof}[#1]}{\end{proof}}
 
% absolute value and norm with correct spacing from the amsmath docs:
\newcommand{\abs}[1]{\left\lvert#1\right\rvert}
\newcommand{\norm}[1]{\left\lVert#1\right\rVert}
% many people define short commands for the set of integers and so on:
\newcommand{\Z}{\mathbb Z}
\newcommand{\N}{\mathbb N}
\newcommand{\Q}{\mathbb Q}
\newcommand{\R}{\mathbb R}
\newcommand{\C}{\mathbb C}
\newcommand{\id}{\mathbbm1} %many use blackboard bold 1 for identity
\newcommand{\mathopf}[1]{{\operatorfont #1}} %better than mathrm for:
\newcommand{\ee}{\mathopf e}                 %Euler e
\newcommand{\ii}{\mathopf i}                 %imaginary (unit) i
\newcommand{\dd}{\mathopf d}                 %differential d
% correct spacing for a set with properties (better than \{x|x\in X\}):
\newcommand{\Meng}[2]{\left\{#1\mathrel{}\middle|\mathrel{}#2\right\}}
% The quotient set (group...) needs different shifting for subscripts;
% we define an optional argument for the negative horizontal space,
% because \! is sometimes not the right amount (before and after /):
\newcommand{\quot}[3][\!]{\mathchoice%
  {\left.\raisebox{.2em}{$\displaystyle#2$}#1\middle/#1%
      \raisebox{-.2em}{$\displaystyle#3$}\right.}%
  {\left.\raisebox{.2em}{$#2$}#1\middle/#1%
      \raisebox{-.2em}{$#3$}\right.}%
  {\left.\raisebox{.08em}{$\scriptstyle#2$}#1\middle/#1%
      \raisebox{-.08em}{$\scriptstyle#3$}\right.}%
  {\left.\raisebox{.05em}{$\scriptscriptstyle#2$}#1\middle/#1%
      \raisebox{-.05em}{$\scriptscriptstyle#3$}\right.}%
}
 
\title{Geschenke}
\author{Osterhase}
\date{24. Dezember 2019}
 
\begin{document}
\maketitle
\section*{Abstract}
Wir suchen Schönheit, so war es auch bei \cite{wei}, und finden sie in
Satz \ref{satz:schoen}.
 
\section{Einleitung}
Die vielleicht schönste Formel ergibt sich aus den Definitionen der
Exponentialreihe und $\pi$.
 
\begin{satz}[Schöne Feststellung]\label{satz:schoen}
  Es gilt
  \begin{equation*}
    \ee^{\ii\pi}+1=0. %Manche mögen Abstand \, vor Satzzeichen.
  \end{equation*}
\end{satz}
 
\begin{beweis}
  \begin{enumerate}
  \item Man erinnert, dass $\pi$ definiert ist als das Doppelte der
    kleinsten positiven Nullstelle von $\cos$.
  \item Die Periodizität von $\sin$ und $\cos$ ist eine übliche
    Übungsaufgabe.
  \item Die Aussage folgt dann direkt aus den Definitionen. \qedhere
    % \qedhere weil der Beweis mit enumerate (equation etc.) endet.
    % (Pakete, die diesen Fall automatisieren, haben auch Nachteile.)
  \end{enumerate}
\end{beweis}
 
\begin{thebibliography}{Wei}
\bibitem[Wei]{wei}
  Weihnachtsmann,
  Geschenke für die Welt,
  Wichtelverlag, 2018.
\end{thebibliography}
\end{document}
